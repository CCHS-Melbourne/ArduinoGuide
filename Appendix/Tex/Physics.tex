\chapter{All the Physics You Will Need}
\label{chap:Physics}
There are two measurable quantities that all electrical circuits share:
\textbf{current} $I$ and \textbf{voltage} $V$.

Current is a way to count the number of electrons flowing past a point in a wire during some period of time, 
for what we will be doing here you can imagine them as tiny tennis balls flying inside the wire.

Voltage is a measure of how much energy each ball has,
thinking of tennis balls again 
you can imagine it as how much it will hurt to be hit by a ball served during the Australian open
verses one gently rolled across the table:
the higher the voltage the more it will hurt.

Notice that the two are independent: 
how many electrons flow past a point is not related to their energy, nor the other way around.

In all electrical circuits 
we use electricity to deliver \textbf{power} $P$ to something.
The power delivered is the current times the voltage: $P = VI$.
Circuit design is about giving every element as much power as it needs,
but not so much that it burns out.

What ``too much'' power means depends on how much power an element converts to heat,
and how efficient at dissipating heat it is.

Knowing only that you and looking at the current voltage characteristics 
you can see why a resistor is a lot more robust than an LED:
a tiny change in the voltage around 
$v_d$
will cause a hugely disproportionate increase in the power delivered, 
quite probably more than enough to burn it out.
For the resistor,
by contrast there,
would have to be a huge increase in the voltage for it to burnout,
since the power delivered to it increases much more evenly.

To deal specifically with circuits you need a further two laws:
Kirchhoff's current and voltage law.

The current law is a way of saying that current can't be created out of nothing:
As much current must leave any junction in a circuit as enters it.
You will also need two laws for dealing with circuits:
Kirchhoff's circuit laws.

 
\def \rot {120}
\begin{center}
\resizebox{0.2\textwidth}{!}{%
\begin{circuitikz} \draw
({cos(\rot - 90)},{sin(\rot - 90)}) to[short, i=$i_a$] (0,0)(0,0) 
to[short, i=$i_b$] ({cos(2*\rot - 90)},{sin(2*\rot - 90)}) 
(0,0) to[short, i=$i_c$] ({cos(3*\rot - 90)},{sin(3*\rot - 90)})
;
\end{circuitikz}
}
\end{center}

In the above we must have $i_a = i_c + i_b$

\colorbox{yellow}{Redo the so it doesn't make your eyes bleed, add a dot in 
the center, fix up the language.}
 
